

%% Additional context
\section{Appendix: Additional Content}
\todo[inline, color=orange]{Do I still need this?}
\paragraph{Scenario X: Complete Response with Historical Data.}

Difference Estimator (using lagged outcome):
\[ \hat{\mu}_{\text{diff}} = \bar{Y}^{(t-1)} + \frac{1}{N}\sum_{i\in S}\frac{y_i - y_i^{(t-1)}}{\pi_i} \]
where $\bar{Y}^{(t-1)} = N^{-1}\sum_{i\in U} y_i^{(t-1)}$ and $y_i^{(t-1)}$ = outcome from previous census.

Model-Assisted Estimator (using auxiliary variables):
\[ \hat{\mu}_{\text{MA}} = \frac{1}{N}\sum_{i\in U}\hat{m}_S(y_i^{(t-1), }, \mathbf{v}_i ) + \frac{1}{N}\sum_{i\in S}\frac{y_i - \hat{m}_S(y_i^{(t-1)}, \mathbf{v}_i )}{\pi_i} \]
where $\mathbf{x}_i$ = auxiliary variables and $\hat{m}_S$ = outcome model fit on sample $S$.
$\mathbf{v}_i = (\mathbf{z}_i, \mathbf{x}_i^{(t-1)})$ = predictor vector ($\mathbf{z}_i$, $\mathbf{x}_i^{(t-1)})$ available for all $i \in U$.

\medskip



\paragraph{Item nonresponse setup.}
Let $r_i=\mathbbm{1}\{y_i \text{ observed}\}\in\{0,1\}$ for $i\in S$ and define
\[
y_i^{\mathrm{obs}}:=r_i\,y_i, \qquad
\tilde y_i := \widehat m(\mathbf{v}_i) \ \ \text{(imputed using } \mathbf{v}_i=[\mathbf{z}_i,\mathbf{x}_i]\text{)},
\]
where $\widehat m$ denotes the chosen imputation method (e.g., regression, hot-deck, MICE) fit on the sample.
We will use the completed value
\[
y_i^\star := r_i\,y_i^{\mathrm{obs}} + (1-r_i)\,\tilde y_i,
\]
so that $y_i^\star=y_i$ when $r_i=1$ and $y_i^\star=\tilde y_i$ when $r_i=0$.

\paragraph{Difference estimator for non-response.}
\todo[inline, color=orange]{Perhaps move to scenario 3?}
With base weights $d_i:=1/\pi_i$ and predictions $f(\mathbf{z}_i)$ available for all $i\in U$,
the PPD estimator using the completed outcomes $y_i^\star$ is
\begin{align}
\widehat{T}_y^{\text{PPD-imp}}
&= \sum_{i\in U} f(\mathbf{z}_i)
 \;+\; \sum_{i\in S} d_i \Bigl( y_i^\star - f(\mathbf{z}_i) \Bigr), \label{eq:PPD-imp-total}\\[3pt]
\widehat{\bar \mu}^{\text{PPD-imp}}
&= \frac{1}{N}\,\widehat{T}_y^{\text{PPD-imp}}. \label{eq:PPD-imp-mean}
\end{align}

Writing the sample correction explicitly,
\begin{align}
\sum_{i\in S} d_i \bigl( y_i^\star - f(\mathbf{z}_i) \bigr)
&= \underbrace{\sum_{i\in S} d_i\, r_i \bigl( y_i^{\mathrm{obs}} - f(\mathbf{z}_i) \bigr)}_{\text{observed-$y$ residuals}}
\;+\;
\underbrace{\sum_{i\in S} d_i\, (1-r_i) \bigl( \tilde y_i - f(\mathbf{z}_i) \bigr)}_{\text{imputed-$y$ residuals}}. \label{eq:obs-imp-split}
\end{align}
\begin{itemize}
  \item If there were no item nonresponse ($r_i\equiv 1$), then $y_i^\star=y_i$ and
  \eqref{eq:PPD-imp-total} reduces to the standard PPD:
  $\widehat{T}_y^{\text{PPD}}=\sum_{i\in U} f(\mathbf{z}_i) + \sum_{i\in S} d_i\,(y_i - f(\mathbf{z}_i))$.
  \item With item nonresponse, \eqref{eq:PPD-imp-total} uses a completed outcome
  $y_i^\star$ formed from $y_i^{\mathrm{obs}}$ (if available) or $\tilde y_i$ (if missing),
  where $\tilde y_i$ is imputed from $\mathbf{v}_i=[\mathbf{z}_i,\mathbf{x}_i]$.
  \item The split \eqref{eq:obs-imp-split} makes clear which part of the correction
  comes from observed vs.\ imputed $y$. Statistical properties then follow from the chosen
  imputation method (e.g., under suitable MAR/regularity, $\tilde y_i$ targets $E[y_i\mid \mathbf{v}_i]$).
\end{itemize}





\paragraph{Difference Estimator with doubly–robust (AIPW) correction for item nonresponse.}
\todo[inline, color=orange]{This will eventually be moved to after scenario 4; TODO: Need to do a literature review. What papers use this type of estimator? }
In the imputation version we replaced missing outcomes by $y_i^\star$ and used
$\widehat T_y^{\text{PPD-imp}}=\sum_{i\in U} f(\mathbf z_i)+\sum_{i\in S} d_i\{y_i^\star-f(\mathbf z_i)\}$.
This is convenient but relies on the outcome model alone to control bias. A standard safeguard is to augment the residual correction with inverse–probability weighting based on a response–propensity model. The resulting estimator is \emph{doubly–robust}: it is consistent if either the outcome model (for $y_i$) or the response model (for $r_i$) is correctly specified.

% We want to be very clear that this is not a new estimator!
This estimator applies the established doubly robust model-calibration framework (e.g., Kim and Rao, 2012) to the historical setting.

\medskip
\noindent\emph{Set–up.} Let $d_i=1/\pi_i$ be the design weight, $\tilde y_i:=\widehat m_s(\mathbf v_i)$ a prediction from an outcome model fit on the sample (using design weights), and
\[
\hat\rho_i:=\widehat{\Pr}(r_i=1\mid \mathbf v_i)
\]
a response–propensity estimate (e.g., weighted logistic/probit or flexible ML classifier fit with weights $d_i$). For numerical stability, bound $\hat\rho_i\in[\varepsilon,1]$ for a small $\varepsilon>0$.


\todo[inline, color=orange]{Need to fix the notation used $\mu$ instead, Also need to standardize $p$ vs. $\rho$..}
\medskip
\noindent\emph{Estimator.} The PPD estimator with doubly–robust (AIPW) residual correction is
\begin{align}
\widehat T_y^{\text{PPD-DR}}
&= \sum_{i\in U} f(\mathbf z_i)
  \;+\; \sum_{i\in S} d_i\Bigl\{\tilde y_i - f(\mathbf z_i)\Bigr\}
  \;+\; \sum_{i\in S_r} \frac{d_i}{\hat\rho_i}\,\Bigl(y_i-\tilde y_i\Bigr),
\label{eq:PPD-DR-total}\\[3pt]
\widehat{\bar Y}^{\text{PPD-DR}}&=\frac{1}{N}\,\widehat T_y^{\text{PPD-DR}}.
\label{eq:PPD-DR-mean}
\end{align}
Equivalently, combining the last two sums,
\[
\widehat T_y^{\text{PPD-DR}}
= \sum_{i\in U} f(\mathbf z_i)
  + \sum_{i\in S} d_i\!\left\{\tilde y_i - f(\mathbf z_i) + \frac{r_i}{\hat\rho_i}\bigl(y_i-\tilde y_i\bigr)\right\},
\]
where the term $y_i$ is only evaluated for respondents ($r_i=1$). Under the convention $y_i^{\mathrm{obs}}=r_i y_i$, one may also write $\tfrac{r_i}{\hat\rho_i}(y_i-\tilde y_i)=\tfrac{r_i}{\hat\rho_i}(y_i^{\mathrm{obs}}-\tilde y_i)$.

\medskip
\noindent\emph{Connections and special cases.}
\begin{itemize}
\item \textbf{Reduction to imputation PPD (Scenario 1).} If $\hat\rho_i\equiv 1$ then
\[
\tilde y_i - f(\mathbf z_i) + r_i\{y_i-\tilde y_i\}
= r_i y_i + (1-r_i)\tilde y_i - f(\mathbf z_i)
= y_i^\star - f(\mathbf z_i),
\]
so $\widehat T_y^{\text{PPD-DR}}=\widehat T_y^{\text{PPD-imp}}$.
\item \textbf{No item nonresponse.} If $r_i\equiv 1$ (hence $\hat\rho_i\equiv 1$), \eqref{eq:PPD-DR-total} reduces to the standard PPD/GREG:
$\sum_{i\in U} f(\mathbf z_i)+\sum_{i\in S} d_i\{y_i-f(\mathbf z_i)\}$.
\item \textbf{Classical AIPW.} If $f\equiv 0$, \eqref{eq:PPD-DR-total} becomes the usual design-weighted AIPW estimator for the total.
\item \textbf{Double robustness.} Under MAR given $\mathbf v_i$, if either (i) the outcome model is mean-correct, $E(y_i\mid \mathbf v_i)=m(\mathbf v_i)$, \emph{or} (ii) the response model is correct, $\Pr(r_i=1\mid \mathbf v_i)=\rho(\mathbf v_i)$, then
\[
E_{S,R}\!\big[\widehat T_y^{\text{PPD-DR}}\big]\;=\;\sum_{i\in U} y_i
\quad\text{(to first order, treating $f(\mathbf z_i)$ as fixed).}
\]
\end{itemize}



\paragraph{Equivalent forms.}
The following expressions are algebraically equivalent ways to write the same total estimator.

\medskip
\noindent\textbf{Form A (compact one–sum):}
\begin{align}
\widehat T_y^{\text{PPD-DR}}
&= \sum_{i\in U} f(\mathbf z_i)
  \;+\; \sum_{i\in S} d_i\Bigl\{
      \underbrace{\tilde y_i - f(\mathbf z_i)}_{\text{PPD/GREG difference}}
      \;+\;
      \underbrace{\frac{r_i}{\hat\rho_i}\bigl(y_i-\tilde y_i\bigr)}_{\text{IPW augmentation}}
    \Bigr\}.
\label{eq:PPD-DR-total-A}
\end{align}

\noindent\textbf{Form B (two–sum version separating respondents):}
\begin{align}
\widehat T_y^{\text{PPD-DR}}
&= \sum_{i\in U} f(\mathbf z_i)
  \;+\; \sum_{i\in S} d_i\bigl\{\tilde y_i - f(\mathbf z_i)\bigr\}
  \;+\; \sum_{i\in S_r} \frac{d_i}{\hat\rho_i}\,\bigl(y_i-\tilde y_i\bigr).
\label{eq:PPD-DR-total-B}
\end{align}

\noindent\textbf{Form C:}
\begin{align}
\widehat T_y^{\text{PPD-DR}}
&= \sum_{i\in U} f(\mathbf z_i)
  \;+\; \sum_{i\in S} d_i\Bigl\{
     \frac{r_i}{\hat\rho_i}\,y_i
     + \Bigl(1-\frac{r_i}{\hat\rho_i}\Bigr)\tilde y_i
     - f(\mathbf z_i)
  \Bigr\}.
\label{eq:PPD-DR-total-C}
\end{align}

\noindent\textbf{Form D (``mixture of residuals''):}
\begin{align}
\widehat T_y^{\text{PPD-DR}}
&= \sum_{i\in U} f(\mathbf z_i)
  \;+\; \sum_{i\in S} d_i\Bigl\{
     \frac{r_i}{\hat\rho_i}\bigl[y_i - f(\mathbf z_i)\bigr]
     + \Bigl(1-\frac{r_i}{\hat\rho_i}\Bigr)\bigl[\tilde y_i - f(\mathbf z_i)\bigr]
  \Bigr\}.
\label{eq:PPD-DR-total-D}
\end{align}

\noindent\textbf{Form E (explicit split into respondents/nonrespondents):}
\begin{align}
\widehat T_y^{\text{PPD-DR}}
&= \sum_{i\in U} f(\mathbf z_i)
  + \sum_{i\in S_r} d_i\Bigl\{\frac{1}{\hat\rho_i}\bigl[y_i-f(\mathbf z_i)\bigr]
      + \Bigl(1-\frac{1}{\hat\rho_i}\Bigr)\bigl[\tilde y_i-f(\mathbf z_i)\bigr]\Bigr\}
  + \sum_{i\in S_m} d_i\bigl[\tilde y_i-f(\mathbf z_i)\bigr].
\label{eq:PPD-DR-total-E}
\end{align}

\medskip
\noindent\textbf{Derivation.}
Starting from Form~\eqref{eq:PPD-DR-total-A}, expand and regroup the terms inside the braces:
\begin{align*}
\tilde y_i - f(\mathbf z_i) + \frac{r_i}{\hat\rho_i}(y_i-\tilde y_i)
&= \frac{r_i}{\hat\rho_i}y_i + \Bigl(1-\frac{r_i}{\hat\rho_i}\Bigr)\tilde y_i - f(\mathbf z_i) \\
&= \frac{r_i}{\hat\rho_i}\bigl[y_i-f(\mathbf z_i)\bigr]
  + \Bigl(1-\frac{r_i}{\hat\rho_i}\Bigr)\bigl[\tilde y_i-f(\mathbf z_i)\bigr],
\end{align*}
which gives Forms~\eqref{eq:PPD-DR-total-C} and \eqref{eq:PPD-DR-total-D}. Splitting the sum in \eqref{eq:PPD-DR-total-C} by respondents ($S_r$) and nonrespondents ($S_m$) yields the explicit Form~\eqref{eq:PPD-DR-total-E}. Form~\eqref{eq:PPD-DR-total-B} is just \eqref{eq:PPD-DR-total-A} with the augmentation written over $S_r$ only.

\medskip
\begin{itemize}
\item The term $\tilde y_i - f(\mathbf z_i)$ is the usual PPD/GREG difference computed for \emph{all} sampled units.
\item The augmentation $\tfrac{r_i}{\hat\rho_i}(y_i-\tilde y_i)$ corrects the sample residuals using inverse–probability weighting and is evaluated only for respondents.
\item In Forms C–E, the coefficients $\tfrac{r_i}{\hat\rho_i}$ and $1-\tfrac{r_i}{\hat\rho_i}$ \emph{sum to one} but are \emph{not} probabilities (for respondents, $1-\tfrac{1}{\hat\rho_i}\le 0$), so these are algebraic mixtures—not convex combinations.
\end{itemize}

\medskip
\noindent\textbf{Sanity checks.}
If $\hat\rho_i\equiv 1$ (ignore nonresponse), all forms collapse to the imputation-based PPD with $y_i^\star=r_i y_i+(1-r_i)\tilde y_i$:
\[
\widehat T_y^{\text{PPD-imp}}
= \sum_{i\in U} f(\mathbf z_i) + \sum_{i\in S} d_i\bigl\{y_i^\star - f(\mathbf z_i)\bigr\}.
\]
If $r_i\equiv 1$ (no item nonresponse), they reduce to the standard PPD/GREG:
\(
\sum_{i\in U} f(\mathbf z_i)+\sum_{i\in S} d_i\{y_i-f(\mathbf z_i)\}.
\)
If $f\equiv 0$, they give the classical design–weighted AIPW total.

